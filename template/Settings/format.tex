%%%%%%%%%%%%%%%%% Оформление ГОСТА%%%%%%%%%%%%%%%%%

% Все параметры указаны в ГОСТЕ на 2021, а именно:

% Шрифт для курсовой Times New Roman, размер – 14 пт.
\setdefaultlanguage[spelling=modern]{russian}
    \setotherlanguage{english}
    
\setmonofont{Times New Roman}
\setmainfont{Times New Roman} 
\setromanfont{Times New Roman} 
\newfontfamily\cyrillicfont{Times New Roman}




% шрифт для URL-ссылок
\urlstyle{same} 

% Междустрочный интервал должен быть равен 1.5 сантиметра.
\linespread{1.5} % междустрочный интервал


% Каждая новая строка должна начинаться с отступа равного 1.25 сантиметра.
\setlength{\parindent}{1.25cm} % отступ для абзаца


% Текст, который является основным содержанием, должен быть выровнен по ширине по умолчанию включен из-за типа документа в main.tex


%%%%%%%%%%%%%%%%%% Дополнения %%%%%%%%%%%%%%%%%%%%%%%%%%%%%%%%%

% Путь до папки с изображениями
\graphicspath{ {./Images/} }

% Внесение titlepage в учёт счётчика страниц
\makeatletter
\renewenvironment{titlepage} {
	\thispagestyle{empty}
}


% Цвет гиперссылок и цитирования
\hypersetup{ 
     colorlinks=true, 
     linkcolor=black, 
     filecolor=blue, 
     citecolor = black,       
     urlcolor=blue, 
}
     
% Настройка сносок по ГОСТу (шрифт 12pt, через один интервал)
\renewcommand{\footnotelayout}{
  \fontsize{12}{14}\selectfont
}
    

% Нумерация рисунков
\counterwithin{figure}{section}

% Нумерация таблиц
\counterwithin{table}{section}

% Нумерация формул
\numberwithin{equation}{section}

% шрифт для листингов с лигатурами
\setmonofont{JetBrains Mono}[
	SizeFeatures={Size=10},
	Contextuals=Alternate
]

\newfontfamily\cyrillicfonttt{JetBrains Mono}[
	SizeFeatures={Size=10}
]

% Перенос текста при переполнении
\emergencystretch=25pt


% настройка подсветки кода и окружения для листингов
%\usemintedstyle{colorful} % делает подсветку для кода
\newenvironment{code}{\captionsetup{type=listing}}{}


% Посмотреть ещё стили можно тут https://www.overleaf.com/learn/latex/Code_Highlighting_with_minted

% Настройка оформления списков по ГОСТу
\setlist{noitemsep} % убираем лишние отступы в списках
\setenumerate{label=\arabic*), ref=\arabic*)} % для числовых списков: 1), 2), 3)...
\setenumerate[2]{label=\alph*), ref=\alph*)} % для вложенных списков: a), b), c)...

% Настройка оформления заголовков по ГОСТу

% Заголовок первого уровня: 15-16pt, все прописные, жирный, по центру, без абзаца, с новой страницы
\titleformat{\section}
{\normalfont\fontsize{15}{18}\bfseries\centering}
{\thesection.}
{1em}
{\MakeUppercase}

% Заголовок второго уровня: 14-15pt, как в предложениях, жирный, по центру, без абзаца
\titleformat{\subsection}
{\normalfont\fontsize{14}{17}\bfseries\centering}
{\thesubsection.}
{1em}
{}

% Заголовок третьего уровня: 14pt, как в предложениях, жирный курсив, по левому краю, без абзаца
\titleformat{\subsubsection}
{\normalfont\fontsize{14}{17}\bfseries\itshape\raggedright}
{\thesubsubsection.}
{1em}
{}

% Добавление пустой строки после заголовков для выделения рубрики
\titlespacing*{\section}{0pt}{*2}{10pt}
\titlespacing*{\subsection}{0pt}{*2}{10pt}
\titlespacing*{\subsubsection}{0pt}{*2}{10pt}

% Оформление подписей к рисункам курсивом, размер шрифта 13pt
\usepackage[font={it,footnotesize}]{caption}
\DeclareCaptionFont{it13pt}{\fontsize{13}{15}\itshape}
\captionsetup[figure]{font=it13pt}

% Оформление подписей к таблицам, размер шрифта 13pt
\captionsetup[table]{font=footnotesize}